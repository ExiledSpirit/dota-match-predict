\documentclass{article}

\usepackage[portuguese]{babel}
\usepackage{inter}

\usepackage[
backend=biber,
style=alphabetic,
sorting=ynt,
style=numeric
]{biblatex}

\addbibresource{bibliografia.bib}

\title{Classificação de Toxicidade em Parágrafos por meio de Aprendizagem de Máquina sobre Word Embeddings}
\author{Lucas Vinicius de Bortoli Santos}

\begin{document}

\pagenumbering{arabic}

\begin{titlepage}
    \begin{center}
        \Huge{Classificação de Toxicidade em Parágrafos por meio de Aprendizagem de Máquina sobre Word Embeddings}
        \vspace{64pt}

        \large{UniSenai PR - Campus Londrina}
        \vspace{48pt}

        \begin{tabular}{l l}
            \textbf{Aluno}                & Lucas Vinicius de Bortoli Santos \\
            \textbf{Aluno}                & Pedro Frasson                    \\
            \textbf{Professor orientador} & Mauricio Noris Freire
        \end{tabular}
    \end{center}
\end{titlepage}

\newpage

\tableofcontents

\newpage
\section{Introdução}

A moderação de comentários e textos é um problema clássico, presente desde a introdução da Internet e do surgimento de comunidades online \cite{DBLP:journals/corr/VaswaniSPUJGKP17}.

\section{Recursos usados}

Durante o desenvolvimento do projeto, foram usados diversos recursos abertos e de alta qualidade. Eles são:

\begin{itemize}
    \item Toolkit scikit-learn, que inclui uma variedade grande de regressores e classificadores disponíveis
    \item Modelo de word embedding \texttt{all-MiniLM-L6-v2}, com um vetor espacial de 384 dimensões e 22,7 milhões de parâmetros
    \item Modelo de word embedding \texttt{stella-en-400M-v5}, com um vetor espacial de 1024 dimensões e 435 milhões de parâmetros
    \item Biblioteca de manipulação de dados Pandas e extensão Data Wranger do Visual Studio Code
\end{itemize}

\newpage
\section{Conclusão}

\newpage

\printbibliography

\end{document}